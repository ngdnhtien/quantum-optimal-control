\documentclass{beamer}
% \usepackage[utf8]{english}
\usepackage{physics}
\usepackage{amsthm}
\usepackage{braket}
\newtheorem{remark}{Remark}
\newcommand{\dcp}{\delta_{CP}}
\renewcommand\bra[1]{{\langle{#1}|}}
\makeatletter
\renewcommand\ket[1]{%
  \@ifnextchar\bra{\k@t{#1}\!}{\k@t{#1}}%
}
\newcommand\k@t[1]{{|{#1}\rangle}}
\DeclareMathAlphabet\mathbfcal{OMS}{cmsy}{b}{n}
\makeatother
\usetheme{Pittsburgh}
\usecolortheme{default}
\AtBeginSection[]
{
  \begin{frame}
    \frametitle{Table of Contents}
    \tableofcontents[currentsection]
  \end{frame}
}

%Information to be included in the title page:
\title[QCVN] %optional
{Leakage reduction in fast superconduting qubit gates via optimal control: Overview}

\author[] % (optional, for multiple authors)
{D. Tien}

\institute[] % (optional)
{
VNU University of Science
}

\date{}


\begin{document}

\frame{\titlepage}
\tableofcontents
\section{Introduction}
\begin{frame}
    \frametitle{Introduction}
    DOI: \href{https://doi.org/10.1038/s41534-020-00346-2}{https://doi.org/10.1038/s41534-020-00346-2}
    \begin{itemize}
      \item Open-loop optimal control theory (gradient/newton-descent) offers means for states and gates realization with high fidelity. 
      \item However, this method produces less accurate results in comparision to ion traps and NMR systems.
      \item As a results, pulse shaping for supercond. qubits requires closed-loop optimal control (finite, continuous space), i.e., direct optimization of the experimental system.
    \end{itemize}
\end{frame}
\section{Setup and system control}
\begin{frame}
  \frametitle{Backend comparison}
  Experiments are carried out on a transmon-type fixed-frequency superconducting qubit. Some numbers:
  \begin{itemize}
    \item transition frequency $\omega_{01}/2\pi=5.11722$ GHz;
    \item anharmonicity $\Delta/2pi=-315.28$ MHz;
    \item $T_1=105\mu$s and $T_2=39\mu$s. 
  \end{itemize}
  Better than armonk ($\sim$us). Not as good as jakarta.
\end{frame}
\begin{frame}
  \frametitle{Realization of $X$ and $Y$ pulses}
  Drive IF signal consists of two control components,
  \begin{align} 
    \Omega = \Omega(t)\exp\{i(\omega_{ssb}t+\phi)\},
  \end{align}
  which is up-converted to the qubit frequency $\omega_{ij}$ and synthesized by AWG. Thus real-time control over phase, frequency, and amplitude. In a frame rotating at the qubit frequency, the transmon Hamiltonian is given by 
  \begin{align} 
    \dfrac{\hat{H}^R}{\hbar} = \Delta\ket{2}\bra{2}+\dfrac{\tilde{\Omega}_x (t)}{2}\sum_{j=1}^2\hat{\sigma}^{x}_{j,j-1}+\dfrac{\tilde{\Omega}_y (t)}{2}\sum_{j=1}^2\hat{\sigma}^{y}_{j,j-1},
  \end{align}
  where $\Omega_{x,y}$ are the drive's IQ-components. Choosing appropriate $\phi$ and $\theta$ results in realization of arbitrary $X$ and $Y$ pulses.
\end{frame}
\begin{frame}
  \frametitle{DRAG first-order correction}
  Since transmons have low anharmonicity, DRAG is employed to suppress leakage out of the computation subspace. To the first-order correction, a Gaussian shaped pulse $\Omega_x(t)=A\exp\{-t^2/(2\sigma^2)\}$ with amplitude $A$ and width $\sigma$, is corrected by
  \begin{align}
    \Omega_{DRAG}(t)=\Omega_x(t)+i\dfrac{\beta}{\Delta}\dfrac{d\Omega_x(t)}{dt},
  \end{align}
  where the imaginary component eliminates the spectral weight of the pulse at $\ket{1}\leftrightarrow\ket{2}$ transition.
\end{frame}
\begin{frame}
  \frametitle{DRAG second-order correction}
  DRAG pulses fail to produce high fidelity state transition when gate duration is lower than ~ 10/$\Delta$\footnote{$\Delta/2\pi=-315.28$ MHz}. To overcome this, higher-order correction terms $\delta_n=a_n+ib_n$ have to be added. This results in a list of piecewise-constant control amplitudes
  \begin{align} 
    \Omega_n = \Omega_{DRAG}(n\Delta t)+\delta_n.
  \end{align}
  The time discretization $\Delta t$ is naturally given by the sampling rate of the AWG generating the pulse envelope. Optimization parameters are amplitude corrections $a_n$ and $b_n$ to the $n$-th sample of $\Omega_{x,y}$. 
\end{frame}
\section{Parameter optimization \& Results}
\begin{frame}
  \frametitle{Pulse parameter optimization}
  \begin{itemize}
    \item Pulse parameters are optimized using the covariance matrix adaptation-evolution strategy (CMA-ES) algorithm. Briefly, $\mathcal{S}^k$ ($\lambda$ sets) are generated, with $k=1,\dots,\lambda$ each characterized by the parametrization of the pulse shape. Each's shape-candidate is evaluated by a cost function, which generates a new set of candidate shapes. 
    \item Randomized benchmarking as cost function ($m$ Clifford gates, $K$ sequences). Clifford gates are constructed by composing $\pm X/2$ and $\pm Y/2$ pulses (!?) based on $\mathcal{S}^k$. 
  \end{itemize}
\end{frame}
\begin{frame}
  \frametitle{Fidelity estimates of optimized short pulses}
  \begin{enumerate}
    \item First round using CMA-ES to calibrate DRAG pulses, $\mathcal{S}=\{A, \beta, \omega_{ssb}\}$.
    \item Next round extending $\mathcal{S}$ to include higher correction orders, namely $a_j$ and $b_j$.
  \end{enumerate} 
  \begin{figure}
    \centering
    \includegraphics[scale=0.18]{sc.png}
    \caption{Optimized DRAG pulse.}
  \end{figure}
\end{frame}
\end{document}